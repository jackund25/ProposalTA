% ================================================================
% BAB V - RENCANA IMPLEMENTASI
% ================================================================
\chapter{Rencana Selanjutnya}

\section{Rencana Implementasi}
Rencana implementasi mencakup tahapan teknis yang diperlukan untuk membangun prototipe sistem prediksi glukosa berbasis \textit{Digital Twin} dan \textit{Retrieval-Augmented Generation} (RAG). Mengingat batasan sumber daya komputasi, strategi implementasi akan difokuskan pada efisiensi model.

\subsection{Persiapan Lingkungan dan Perangkat Pengembangan}
Pengembangan sistem dilakukan menggunakan perangkat keras laptop ASUS VivoBook (M3400QA) dengan spesifikasi sebagai berikut:
\begin{itemize}
    \item \textbf{Prosesor}: AMD Ryzen™ 5 5600H (6 Cores, 12 Threads, up to 4.2 GHz)
    \item \textbf{Grafis}: AMD Radeon™ Graphics (Integrated)
    \item \textbf{Memori (RAM)}: 16 GB DDR4
    \item \textbf{Penyimpanan}: 512 GB SSD NVMe
    \item \textbf{Sistem Operasi}: Windows 11 Home Single Language 64-bit
\end{itemize}

Perangkat lunak dan alat pengembangan yang digunakan meliputi:
\begin{itemize}
    \item \textbf{IDE}: Visual Studio Code (VS Code) dengan ekstensi Python dan Jupyter.
    \item \textbf{Bahasa Pemrograman}: Python 3.10+.
    \item \textbf{Version Control}: Git dan GitHub untuk manajemen kode sumber dan kolaborasi.
    \item \textbf{Environment Manager}: Anaconda atau Python \textit{venv} untuk isolasi dependensi.
\end{itemize}

Mengingat perangkat tidak memiliki akselerator grafis diskrit (GPU NVIDIA CUDA), strategi komputasi akan disesuaikan:
\begin{enumerate}
    \item \textbf{Pelatihan LSTM}: Dilakukan menggunakan komputasi CPU (Ryzen 5) yang sangat memadai untuk dataset numerik berukuran sedang.
    \item \textbf{Inferensi RAG}: Menggunakan model bahasa yang telah dikompresi (\textit{Quantized Model} format GGUF 4-bit) agar dapat berjalan efisien di RAM sistem, atau memanfaatkan layanan \textit{Cloud API} (seperti Groq/HuggingFace) untuk menghindari beban berlebih pada perangkat lokal.
\end{enumerate}

\subsection{Persiapan Data}
Tahap ini mencakup:
\begin{itemize}
    \item Penyiapan repositori data di GitHub.
    \item Penentuan skema data untuk \textit{Logbook} Harian (variabel: glukosa, karbohidrat, insulin, aktivitas, stres).
    \item Pembersihan (\textit{cleaning}) dataset publik (OhioT1DM) untuk keperluan \textit{pre-training} model.
    \item Implementasi teknik \textit{sliding window} untuk mengubah data runtun waktu menjadi format \textit{supervised learning}.
\end{itemize}

\subsection{Pengembangan Model Prediksi}
Tahap pemodelan mencakup:
\begin{enumerate}
    \item Pembuatan model \textit{baseline} (Random Forest) untuk tolok ukur performa.
    \item Pengembangan model \textit{Deep Learning} arsitektur LSTM dengan mekanisme \textit{Attention}.
    \item \textit{Hyperparameter tuning} untuk mengoptimalkan \textit{learning rate} dan jumlah \textit{layer}.
    \item Evaluasi model menggunakan data uji untuk memastikan target RMSE $\leq$ 25 mg/dL tercapai.
\end{enumerate}

\subsection{Pembangunan Digital Twin}
Digital Twin dibangun sebagai modul perangkat lunak yang mereplikasi logika metabolisme pasien. Kegiatan meliputi:
\begin{itemize}
    \item Perancangan struktur data \textit{JSON} untuk menyimpan \textit{state} digital pasien.
    \item Implementasi logika pembaruan \textit{state} otomatis berdasarkan output prediksi LSTM.
    \item Pengembangan algoritma simulasi untuk fitur \textit{What-If Analysis} (misal: kalkulasi dampak jika variabel stres diturunkan 20\%).
\end{itemize}

\subsection{Integrasi Retrieval-Augmented Generation (RAG)}
Tahapan ini meliputi:
\begin{itemize}
    \item Penyusunan basis pengetahuan (\textit{Knowledge Base}) dari panduan klinis diabetes terpercaya.
    \item Implementasi \textit{vector database} lokal (menggunakan FAISS atau ChromaDB) untuk penyimpanan dokumen.
    \item Pengembangan fungsi \textit{Retriever} untuk pencarian semantik.
    \item Integrasi dengan LLM (Mistral-7B versi GGUF atau API) untuk menghasilkan narasi penjelasan.
\end{itemize}

\subsection{Pengembangan Antarmuka dan Visualisasi}
Antarmuka pengguna (\textit{User Interface}) sederhana berbasis web (menggunakan Streamlit atau Flask) akan dikembangkan untuk menampilkan:
\begin{itemize}
    \item Formulir input log harian.
    \item Grafik tren glukosa aktual vs prediksi.
    \item Panel simulasi interaktif.
    \item Kotak penjelasan rekomendasi dari RAG.
\end{itemize}

\subsection{Dokumentasi dan Penyusunan Laporan}
Seluruh kode program akan didokumentasikan di GitHub, dan hasil eksperimen akan disusun menjadi laporan Tugas Akhir.

\section{Desain Pengujian dan Evaluasi}
Pengujian sistem dilakukan melalui beberapa tahap:

\subsection{Verifikasi Sistem}
Verifikasi memastikan komponen berjalan sesuai desain:
\begin{itemize}
    \item \textbf{Unit Testing}: Menguji fungsi preprocessing dan kalkulasi metrik.
    \item \textbf{Integration Testing}: Memastikan data mengalir lancar dari Input Log $\rightarrow$ LSTM $\rightarrow$ RAG.
\end{itemize}

\subsection{Validasi Model Prediksi}
Evaluasi akurasi model kuantitatif menggunakan metrik:
\begin{itemize}
    \item \textit{Root Mean Square Error} (RMSE).
    \item \textit{Mean Absolute Error} (MAE).
    \item \textit{Clarke Error Grid Analysis} (untuk validasi keamanan klinis).
\end{itemize}

\subsection{Evaluasi Integrasi Digital Twin dan RAG}
Evaluasi kualitatif dilakukan terhadap:
\begin{itemize}
    \item Kecepatan respons simulasi skenario.
    \item Relevansi dokumen medis yang diambil oleh \textit{Retriever}.
    \item Keterbacaan dan kejelasan rekomendasi teks yang dihasilkan oleh RAG.
\end{itemize}

\section{Analisis Risiko dan Mitigasi}
Tabel V.1 merangkum risiko teknis yang mungkin terjadi dan strategi mitigasinya.

\begin{table}[H]
    \centering
    \caption{Analisis Risiko dan Mitigasi}
    \label{tbl:analisis-risiko}
    \begin{tabular}{|p{3.5cm}|p{4.5cm}|p{5cm}|}
    \hline
    \textbf{Risiko} & \textbf{Dampak} & \textbf{Mitigasi} \\ \hline
    
    Keterbatasan komputasi lokal (Tanpa GPU Diskrit) & 
    Inferensi LLM lokal berjalan lambat atau membebani memori sistem (RAM). & 
    1. Menggunakan model terkuantisasi (4-bit GGUF) yang ringan untuk CPU.\newline
    2. Menggunakan layanan Cloud API untuk modul Generator. \\ \hline
    
    Data input manual tidak konsisten (bolong-bolong) & 
    Akurasi prediksi LSTM menurun karena \textit{missing values}. & 
    Implementasi teknik imputasi data (interpolasi) pada tahap \textit{preprocessing}. \\ \hline
    
    \textit{Overfitting} pada data latih yang kecil & 
    Model gagal memprediksi data baru dengan baik. & 
    Penerapan teknik regularisasi (\textit{Dropout}) dan validasi silang (\textit{Cross-validation}). \\ \hline
    
    Halusinasi pada RAG & 
    Sistem memberikan saran medis yang salah atau tidak relevan. & 
    Membatasi konteks dokumen hanya pada sumber terpercaya (Panduan PERKENI/ADA) dan menurunkan parameter \textit{temperature} LLM. \\ \hline
    
    \end{tabular}
\end{table}

\section{Timeline Pengerjaan}
Timeline disusun berdasarkan periode Januari-Mei 2026. Jadwal rinci disajikan pada Gambar V.1.

\begin{figure}[H]
    \centering
    \includegraphics[width=0.95\textwidth]{image/gantt-chart.png}
    \caption{Timeline Pelaksanaan Tugas Akhir (Januari-Mei 2026).}
    \label{gambar:timeline}
\end{figure}