% ==========================================
% BAB V RENCANA SELANJUTNYA
% ==========================================
\chapter{Rencana Selanjutnya}

\section{Rencana Implementasi}

Rencana implementasi mencakup tahapan teknis yang diperlukan untuk membangun prototipe sistem prediksi glukosa berbasis Digital Twin dan Retrieval-Augmented Generation (RAG). Secara umum, tahapan implementasi meliputi:

\subsection{Persiapan Lingkungan dan Perangkat Pengembangan}

Lingkungan pengembangan menggunakan perangkat keras utama berupa \textbf{Asus Vivobook Pro 14 OLED} dengan spesifikasi:

\begin{itemize}
    \item Prosesor: AMD Ryzen 7 5800H / Intel Core i5/i7 (varian seri)
    \item GPU: NVIDIA GeForce RTX 3050 / 3050Ti
    \item RAM: 16 GB
    \item Storage: 512 GB SSD NVMe
    \item Sistem Operasi: Windows 11
    \item Lingkungan kerja: VS Code, Python Virtual Environment
\end{itemize}

Perangkat tersebut memadai untuk kebutuhan pelatihan model machine learning skala menengah, eksperimen deep learning dasar, serta pemrosesan data.

\subsection{Persiapan Data}

Tahap ini mencakup:

\begin{itemize}
    \item Penentuan format data input dan struktur variabel.
    \item Eksplorasi karakteristik data, identifikasi missing values, dan pemeriksaan distribusi.
    \item Proses pembersihan data (cleaning) dan normalisasi.
    \item Penyusunan sequence atau time-window untuk mendukung pemodelan deret waktu.
\end{itemize}

\subsection{Pengembangan Model Prediksi}

Tahap pemodelan mencakup:

\begin{enumerate}
    \item Pembuatan model baseline (misal: Random Forest, Support Vector Regression, MLP).
    \item Pengembangan model berbasis \textit{deep learning} seperti LSTM atau GRU.
    \item Tuning hiperparameter untuk meningkatkan performa model.
    \item Pemilihan model terbaik berdasarkan metrik evaluasi.
\end{enumerate}

\subsection{Pembangunan Digital Twin}

Digital Twin digunakan sebagai representasi keadaan pasien secara virtual. Kegiatan yang dilakukan antara lain:

\begin{itemize}
    \item Perancangan state digital yang mencerminkan kondisi glukosa.
    \item Implementasi mekanisme pembaruan state berdasarkan prediksi model.
    \item Simulasi skenario perubahan variabel seperti insulin, karbohidrat, dan aktivitas.
\end{itemize}

\subsection{Integrasi Retrieval-Augmented Generation (RAG)}

Tahapan ini meliputi:

\begin{itemize}
    \item Penyusunan basis pengetahuan eksternal.
    \item Implementasi \textit{embedding} dan \textit{vector store}.
    \item Pengembangan modul \textit{retriever} dan \textit{generator}.
    \item Integrasi modul RAG dengan Digital Twin untuk memberikan penjelasan berbasis bukti.
\end{itemize}

\subsection{Pengembangan Antarmuka dan Visualisasi}

Antarmuka sederhana akan dikembangkan untuk menampilkan:

\begin{itemize}
    \item Grafik glukosa aktual dan hasil prediksi.
    \item Simulasi Digital Twin.
    \item Penjelasan prediktif yang dihasilkan oleh RAG.
\end{itemize}

\subsection{Dokumentasi dan Penyusunan Laporan}

Seluruh proses, hasil eksperimen, serta evaluasi sistem akan didokumentasikan untuk penyusunan laporan Tugas Akhir.

\section{Desain Pengujian dan Evaluasi}

Pengujian sistem dilakukan melalui beberapa tahap:

\subsection{Verifikasi Sistem}

Verifikasi mencakup:

\begin{itemize}
    \item Konsistensi preprocessing.
    \item Stabilitas model selama pelatihan.
    \item Kesesuaian pembaruan state pada Digital Twin.
    \item Relevansi dokumen hasil retrieval pada modul RAG.
\end{itemize}

\subsection{Validasi Model Prediksi}

Evaluasi dilakukan menggunakan metrik:

\begin{itemize}
    \item Root Mean Square Error (RMSE)
    \item Mean Absolute Error (MAE)
    \item Clarke Error Grid Analysis
\end{itemize}

\subsection{Evaluasi Integrasi Digital Twin dan RAG}

Evaluasi dilakukan terhadap:

\begin{itemize}
    \item Konsistensi pembaruan state Digital Twin.
    \item Akurasi konteks penjelasan yang diberikan RAG.
    \item Keterbacaan insight bagi pengguna.
\end{itemize}

\section{Analisis Risiko dan Mitigasi}

\begin{table}[H]
\centering
\begin{tabular}{|p{3.5cm}|p{4cm}|p{5cm}|}
\hline
\textbf{Risiko} & \textbf{Dampak} & \textbf{Mitigasi} \\
\hline
Kualitas data tidak konsisten & Model tidak stabil & Pembersihan data, normalisasi, pemeriksaan manual \\ \hline
Overfitting & Generalisasi buruk & Regularisasi, dropout, early stopping \\ \hline
Keterbatasan perangkat komputasi & Waktu pelatihan lebih lama & Penggunaan batch kecil, optimasi model, opsi GPU cloud \\ \hline
RAG menghasilkan insight tidak relevan & Penjelasan menyesatkan & Perbaikan knowledge base, evaluasi retrieval \\ \hline
Integrasi DT kompleks & Hambatan implementasi & Pendekatan bertahap, mulai dari versi minimal \\ \hline
\end{tabular}
\caption{Analisis Risiko dan Mitigasi}
\end{table}

\section{Timeline Pengerjaan}

Timeline disusun berdasarkan periode Januari–Mei 2026, dengan pembagian per minggu. Format tabel mengikuti struktur jadwal pada laporan kerja praktik sebelumnya.

\subsection*{Tabel Timeline Pelaksanaan Tugas Akhir (Januari–Mei 2026)}

\begin{figure}[H]
    \centering
    \includegraphics[width=0.95\textwidth]{image/gantt-chart.png}
    \caption{Timeline Pelaksanaan Tugas Akhir (Januari–Mei 2026)}
    \label{gambar:gantt-chart}
\end{figure}

\clearpage
