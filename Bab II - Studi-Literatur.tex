% ==========================================
% BAB II STUDI LITERATUR
% ==========================================
\chapter{STUDI LITERATUR}
\label{chap:studi-literatur}

\section{Konsep Digital Twin}

Konsep \textit{Digital Twin} (DT) pertama kali diperkenalkan oleh \textcite{grieves2017digital} sebagai bagian dari paradigma \textit{Product Lifecycle Management} (PLM). DT didefinisikan sebagai representasi digital dari suatu entitas fisik yang secara dinamis diperbarui melalui aliran data dua arah antara dunia nyata (\textit{real space}) dan dunia virtual (\textit{virtual space}). Melalui keterhubungan tersebut, sistem DT dapat melakukan pemantauan, simulasi, serta prediksi terhadap perilaku dan kondisi objek fisik secara berkelanjutan.

Secara umum, arsitektur DT terdiri dari tiga komponen utama:
\begin{enumerate}
    \item \textbf{Physical Entity}, yaitu objek nyata yang menjadi sumber data (contohnya mesin industri, kendaraan, atau pasien dalam konteks medis);
    \item \textbf{Digital Representation}, yaitu model virtual yang merepresentasikan perilaku dan karakteristik objek fisik;
    \item \textbf{Data Connection Layer}, yaitu lapisan penghubung yang mengintegrasikan aliran data dua arah antara model digital dan entitas fisik, baik untuk pembaruan kondisi maupun umpan balik hasil simulasi.
\end{enumerate}

Contoh arsitektur dasar DT dapat dilihat pada Gambar~\ref{gambar:dt-konsep}. Diagram tersebut merupakan model konseptual awal yang dikembangkan oleh Grieves untuk menjelaskan hubungan dua ruang antara sistem fisik dan sistem virtual pada konteks PLM.

\begin{figure}[h]
    \centering
    \captionsetup{justification=centering}
    \includegraphics[width=0.8\textwidth]{image/dt_concept_grieves2017.png}
    \caption{Model konseptual \textit{Digital Twin} yang menunjukkan keterhubungan antara ruang fisik dan ruang virtual. Diadaptasi dari \textcite{grieves2017digital}.}
    \label{gambar:dt-konsep}
\end{figure}

Gambar~\ref{gambar:dt-konsep} menampilkan dua ruang utama, yaitu \textit{Real Space} dan \textit{Virtual Space}. Pada bagian \textit{Real Space}, terdapat entitas fisik yang menghasilkan data melalui sensor dan sistem pengukuran. Data tersebut dikirim ke \textit{Virtual Space} untuk memperbarui model digital secara berkelanjutan. Sebaliknya, hasil simulasi dan analisis dari model virtual dikirim kembali ke sistem fisik sebagai informasi atau proses kontrol. Siklus pertukaran data ini menciptakan hubungan dua arah yang memungkinkan DT berfungsi sebagai sistem adaptif dan prediktif \autocite{grieves2017digital}.

Menurut \textcite{Bruynseels2018}, keunggulan utama DT terletak pada kemampuannya untuk menghasilkan \textit{living model}, yakni model yang tidak hanya merepresentasikan kondisi statis suatu sistem, tetapi juga mampu berevolusi mengikuti dinamika data aktual. Hal ini menjadikan DT sangat potensial dalam pengembangan sistem prediktif di berbagai bidang, termasuk manufaktur, pertahanan, energi, dan kesehatan.

Lebih lanjut, \textcite{Cappon2024} menjelaskan bahwa arsitektur DT modern telah berevolusi dari model PLM klasik menjadi sistem terintegrasi multi-sumber data. Model ini tidak hanya menerima input dari sensor fisik, tetapi juga dari basis pengetahuan eksternal seperti rekam medis elektronik, data \textit{wearable devices}, dan sistem informasi klinis. Dengan kemampuan integrasi ini, DT berfungsi sebagai platform analitik cerdas yang mampu melakukan prediksi, optimasi, dan personalisasi secara simultan.

Dalam konteks penelitian ini, pemahaman terhadap konsep dan arsitektur dasar DT menjadi landasan untuk mengembangkan model \textit{Digital Twin} yang dapat merepresentasikan kondisi fisiologis pasien diabetes tipe~2 secara virtual. Model digital tersebut akan digunakan untuk proses simulasi dan prediksi kadar glukosa darah berdasarkan data simulatif, sebelum nantinya dikembangkan lebih lanjut melalui pendekatan \textit{Retrieval-Augmented Generation (RAG)} di bab berikutnya.

\section{Digital Twin dalam Bidang Kesehatan}

Konsep \textit{Digital Twin} telah berkembang pesat dalam bidang kesehatan dan kedokteran presisi selama satu dekade terakhir. Menurut \textcite{Bruynseels2018}, penerapan DT di bidang medis membuka peluang untuk mengembangkan model digital pasien yang mampu merepresentasikan kondisi fisiologis individu secara dinamis. Model tersebut dapat digunakan untuk melakukan prediksi hasil klinis, simulasi pengobatan, hingga personalisasi terapi berdasarkan karakteristik spesifik pasien.

Salah satu penerapan paling menonjol adalah pada bidang manajemen penyakit kronis seperti diabetes. \textcite{Cappon2024} dalam tinjauan sistematisnya menunjukkan bahwa DT digunakan untuk mensimulasikan perilaku metabolik tubuh dan memprediksi kadar glukosa darah berdasarkan data historis pasien. Sistem ini berfungsi sebagai \textit{virtual replica} dari tubuh pasien yang beroperasi secara paralel dengan sistem nyata untuk memberikan wawasan tentang status kesehatan serta potensi respon terhadap terapi.

\begin{figure}[h]
    \centering
    \captionsetup{justification=centering}
    \includegraphics[width=0.85\textwidth]{image/dt_healthcare_cappon2024.png}
    \caption{Arsitektur umum \textit{Digital Twin} dalam bidang kesehatan yang mengintegrasikan data dari perangkat medis, sensor, dan catatan kesehatan elektronik. Diadaptasi dari \textcite{Cappon2024}.}
    \label{gambar:dt-healthcare}
\end{figure}

Gambar~\ref{gambar:dt-healthcare} menampilkan arsitektur umum \textit{Digital Twin} dalam bidang kesehatan sebagaimana dijelaskan oleh \textcite{Cappon2024}. Diagram tersebut memperlihatkan bahwa DT medis terdiri atas beberapa lapisan integrasi data, meliputi:
\begin{enumerate}
    \item \textbf{Sumber Data Fisik}, yang mencakup perangkat medis seperti \textit{Continuous Glucose Monitoring} (CGM), sensor aktivitas, dan perangkat \textit{wearable};
    \item \textbf{Lapisan Data dan Analitik}, tempat data klinis, fisiologis, dan perilaku pasien diproses serta diintegrasikan ke dalam model digital;
    \item \textbf{Model Digital Pasien (Patient Twin)}, yaitu representasi virtual pasien yang digunakan untuk prediksi, simulasi, dan evaluasi terapi; serta
    \item \textbf{Interface Klinik}, yang memungkinkan dokter atau sistem pendukung keputusan memanfaatkan hasil simulasi untuk tindakan medis.
\end{enumerate}

Melalui arsitektur tersebut, sistem DT memungkinkan analisis prediktif yang lebih akurat karena dapat memperhitungkan variabilitas fisiologis antar individu. \textcite{Rad2024} bahkan mengusulkan framework \textit{Patient-Centric Digital Twin} berbasis \textit{Personal Health Knowledge Graph (PHKG)} yang mampu mengintegrasikan berbagai sumber data klinis dengan standar interoperabilitas HL7 FHIR. Framework tersebut terbukti mampu meningkatkan akurasi prediksi kadar glukosa dengan nilai \textit{Root Mean Square Error (RMSE)} sebesar 19{,}83 mg/dL.

\textcite{Zhang2024} memperluas konsep ini dengan mengintegrasikan data multi-omics ke dalam model DT untuk memprediksi progresi penyakit \textit{Type 2 Diabetes Mellitus} (T2DM). Pendekatan tersebut menunjukkan potensi besar DT dalam mendukung \textit{personalized medicine} dengan memanfaatkan data fisiologis, genetik, dan perilaku secara holistik.

Meskipun hasilnya menjanjikan, sebagian besar penerapan DT di bidang kesehatan masih menghadapi kendala berupa kebutuhan akan infrastruktur teknologi tinggi, biaya perangkat medis, dan ketersediaan data real-time yang memadai \autocite{Cappon2024, Rad2024}. Tantangan ini semakin besar di negara berkembang, di mana infrastruktur digital dan rekam medis elektronik belum merata. Oleh karena itu, adaptasi konsep DT dengan pendekatan berbasis data simulatif dan kecerdasan buatan menjadi alternatif yang relevan untuk konteks sistem kesehatan seperti di Indonesia. 

Selain pendekatan berbasis model fisiologis yang kompleks, perkembangan terkini juga menunjukkan tren positif pada pendekatan \textit{Data-Driven Digital Twin}. Berbeda dengan model mekanistik yang memerlukan parameter biologis mendalam, pendekatan berbasis data memanfaatkan algoritma pembelajaran mesin untuk mempelajari pola dinamika sistem dari data historis \autocite{Woldaregay2019}. Pendekatan ini dinilai lebih fleksibel dan \textit{feasible} untuk diterapkan pada kondisi dengan keterbatasan infrastruktur sensor karena dapat beroperasi menggunakan data observasional tanpa mengurangi esensi kemampuan prediktif dan simulasi dari konsep \textit{Digital Twin} itu sendiri.

\section{Manajemen Diabetes dan Pendekatan Digital}

\textit{Diabetes Mellitus} (DM) merupakan penyakit metabolik kronis yang ditandai oleh meningkatnya kadar glukosa darah akibat gangguan pada sekresi atau kerja insulin. Bentuk yang paling umum adalah \textit{Diabetes Mellitus Tipe 2} (T2DM) yang menyumbang lebih dari 90\% kasus diabetes di dunia dan di Indonesia \autocite{WHO2023}. Penyakit ini berhubungan erat dengan resistensi insulin serta penurunan sensitivitas sel terhadap glukosa, sehingga pengelolaan kadar glukosa menjadi aspek kunci dalam pencegahan komplikasi kronis seperti penyakit jantung, gagal ginjal, neuropati, dan kebutaan.

Pendekatan konvensional dalam manajemen T2DM melibatkan pemantauan kadar glukosa darah secara berkala, pengaturan pola makan, aktivitas fisik, serta pemberian terapi insulin atau obat oral. Namun, strategi ini seringkali bersifat reaktif, yaitu penyesuaian dilakukan setelah kadar glukosa menyimpang dari rentang normal. Pendekatan seperti ini belum sepenuhnya mampu mencegah fluktuasi glukosa ekstrem yang berisiko menimbulkan komplikasi jangka panjang \autocite{Alkaff2021}.

Untuk meningkatkan efektivitas pengelolaan diabetes, berbagai teknologi digital telah dikembangkan, antara lain \textit{Continuous Glucose Monitoring} (CGM) dan \textit{insulin pump} yang memungkinkan pemantauan dan penyesuaian terapi secara real-time \autocite{Battelino2019}. Sistem ini mengubah paradigma pengelolaan diabetes dari berbasis manual menjadi berbasis data (\textit{data-driven management}). Selain itu, penggunaan aplikasi mobile dan perangkat \textit{wearable} juga memungkinkan pasien untuk melacak pola makan, aktivitas fisik, serta kadar glukosa harian secara otomatis.

\begin{figure}[H]
    \centering
    \captionsetup{justification=centering}
    \includegraphics[width=0.85\textwidth]{image/diabetes_management_cgm_battelino2019.png}
    \caption{Contoh sistem pemantauan kadar glukosa darah berkelanjutan (\textit{Continuous Glucose Monitoring}) yang digunakan dalam manajemen diabetes. Diadaptasi dari \textcite{Battelino2019}.}
    \label{gambar:diabetes-management}
\end{figure}

Gambar~\ref{gambar:diabetes-management} memperlihatkan contoh sistem \textit{Continuous Glucose Monitoring} (CGM) yang memungkinkan akuisisi data glukosa secara kontinu dari sensor subkutan. Data dari CGM dikirim ke perangkat penerima atau aplikasi seluler untuk analisis dan visualisasi pola glukosa pasien. Informasi ini dapat digunakan untuk menyesuaikan dosis insulin secara otomatis atau semi-otomatis, sehingga membantu menjaga kadar glukosa dalam rentang aman \autocite{Battelino2019}. 

Selanjutnya, kemajuan dalam bidang \textit{machine learning} memungkinkan analisis pola data glukosa dalam jangka panjang untuk tujuan prediktif. \textcite{Woldaregay2019} mengembangkan model pembelajaran mesin yang mampu memprediksi fluktuasi glukosa darah jangka pendek berdasarkan data historis pasien. Pendekatan ini menunjukkan bahwa model berbasis data dapat membantu pasien mengantisipasi episode hipoglikemia atau hiperglikemia sebelum terjadi.

Meskipun teknologi tersebut memberikan kemajuan signifikan, penerapannya di Indonesia masih terbatas. Studi oleh \textcite{Ramadaniati2024} menunjukkan bahwa harga perangkat CGM di Indonesia relatif tinggi, dengan biaya yang dapat mencapai satu hingga dua bulan gaji rata-rata pasien. Kondisi ini menimbulkan tantangan dalam pemerataan akses terhadap teknologi pemantauan digital, terutama di fasilitas kesehatan primer.

Terlepas dari keunggulan data granularitas tinggi yang ditawarkan CGM, metode Pemantauan Gula Darah Mandiri (PGDM) atau \textit{Self-Monitoring of Blood Glucose} (SMBG) tetap menjadi standar emas perawatan di banyak negara berkembang karena faktor biaya dan ketersediaan \autocite{Ramadaniati2024}. Studi menunjukkan bahwa data SMBG yang dicatat secara terstruktur dan konsisten---metode yang dikenal sebagai \textit{Ecological Momentary Assessment} (EMA)---tetap memiliki validitas klinis yang tinggi untuk memprediksi tren glikemik harian. Oleh karena itu, pengembangan sistem cerdas yang mampu mengoptimalkan data SMBG menjadi solusi yang lebih inklusif dan realistis untuk konteks demografi Indonesia dibandingkan ketergantungan penuh pada sensor CGM.

Berangkat dari kondisi tersebut, diperlukan pendekatan alternatif yang terjangkau namun tetap memiliki kemampuan prediktif tinggi. Salah satu solusi yang potensial adalah penggunaan model \textit{Digital Twin} berbasis data simulatif, yang dapat menggantikan kebutuhan perangkat real-time melalui pemanfaatan data historis atau sintetis untuk pelatihan model prediksi glukosa. Pendekatan ini membuka peluang bagi pengembangan sistem prediktif adaptif yang relevan untuk konteks sumber daya terbatas seperti di Indonesia.

\section{Faktor-Faktor yang Mempengaruhi Kadar Gula Darah}

Kadar glukosa darah merupakan variabel fisiologis yang sangat dinamis dan dipengaruhi oleh interaksi kompleks berbagai faktor internal maupun eksternal. Pemahaman mendalam mengenai faktor-faktor ini menjadi landasan penting dalam pengembangan model prediksi yang akurat. Secara umum, terdapat tiga variabel utama yang berkontribusi signifikan terhadap fluktuasi glukosa harian, yaitu asupan nutrisi, aktivitas fisik, dan kondisi psikologis.

\subsection{Asupan Nutrisi dan Karbohidrat}
Asupan makanan, khususnya karbohidrat, memiliki dampak langsung dan paling signifikan terhadap kenaikan kadar glukosa darah pascamakan (\textit{postprandial}). Karbohidrat yang dikonsumsi akan dihidrolisis menjadi glukosa dan diserap ke dalam aliran darah, menyebabkan lonjakan kadar gula darah dalam waktu singkat. Pada pasien T2DM, gangguan respons insulin menyebabkan tubuh gagal mengembalikan kadar glukosa ke rentang normal dengan cepat \autocite{ADA2023}. Oleh karena itu, pencatatan jumlah asupan karbohidrat (\textit{carbohydrate counting}) menjadi variabel input yang krusial dalam algoritma prediksi glukosa \autocite{Woldaregay2019}.

\subsection{Aktivitas Fisik}
Aktivitas fisik berperan sebagai mekanisme regulasi alami yang membantu menurunkan kadar glukosa darah. Saat melakukan aktivitas fisik atau olahraga, otot rangka meningkatkan pengambilan glukosa dari darah untuk digunakan sebagai energi, bahkan tanpa memerlukan peningkatan insulin yang signifikan. Selain itu, aktivitas fisik rutin dapat meningkatkan sensitivitas insulin jangka panjang \autocite{Colberg2016}. Dalam konteks pemodelan digital twin, data durasi dan intensitas aktivitas fisik digunakan untuk mensimulasikan laju penurunan glukosa (\textit{glucose clearance rate}).

\subsection{Faktor Psikologis dan Stres}
Selain faktor fisik, kondisi psikologis seperti stres memiliki pengaruh yang sering diabaikan namun signifikan terhadap manajemen diabetes. Stres psikologis mengaktifkan sumbu \textit{Hypothalamic-Pituitary-Adrenal} (HPA) dan sistem saraf simpatis, yang memicu pelepasan hormon stres seperti kortisol dan adrenalin (katekolamin) \autocite{Hackett2017}. 

Secara fisiologis, hormon-hormon ini bekerja antagonis terhadap insulin dengan cara:
\begin{itemize}
    \item Merangsang glikogenolisis (pemecahan glikogen menjadi glukosa) di hati.
    \item Meningkatkan glukoneogenesis (pembentukan glukosa baru).
    \item Meningkatkan resistensi insulin pada jaringan perifer.
\end{itemize}
Akibatnya, episode stres yang tinggi dapat menyebabkan hiperglikemia meskipun pasien telah menjaga pola makan dan obat-obatan. Oleh karena itu, integrasi variabel tingkat stres ke dalam sistem prediksi sangat diperlukan untuk meningkatkan akurasi model pada kondisi dunia nyata.

\section{Metode Machine Learning untuk Prediksi Kadar Glukosa}

Prediksi kadar glukosa darah merupakan salah satu tantangan utama dalam manajemen \textit{Diabetes Mellitus}. Kompleksitas hubungan antara berbagai variabel fisiologis seperti asupan karbohidrat, dosis insulin, aktivitas fisik, dan stres membuat pendekatan konvensional berbasis model matematis sulit mencapai akurasi tinggi. Dalam konteks ini, pendekatan \textit{machine learning} (ML) muncul sebagai solusi potensial untuk menangkap pola non-linear yang tidak dapat dimodelkan secara eksplisit.

Menurut \textcite{Woldaregay2019}, model pembelajaran mesin dapat mempelajari hubungan kompleks antara variabel fisiologis dari data historis pasien. Pendekatan ini melibatkan pelatihan algoritma menggunakan dataset yang berisi pasangan masukan (misalnya kadar glukosa sebelumnya, dosis insulin, asupan makanan) dan keluaran (kadar glukosa berikutnya). Setelah dilatih, model dapat digunakan untuk memprediksi kadar glukosa masa depan berdasarkan pola yang telah dipelajari.

Beberapa algoritma yang umum digunakan dalam prediksi glukosa antara lain \textit{Random Forest (RF)}, \textit{Support Vector Regression (SVR)}, dan model berbasis jaringan saraf dalam seperti \textit{Long Short-Term Memory (LSTM)} \autocite{li2022time}. Model LSTM populer karena kemampuannya memproses data deret waktu dan mempertahankan informasi jangka panjang melalui mekanisme \textit{memory cell} dan \textit{gating}. 

\begin{figure}[H]
    \centering
    \captionsetup{justification=centering}
    \includegraphics[width=0.95\textwidth]{image/lstm_architecture.png}
    \caption{Arsitektur model \textit{Long Short-Term Memory (LSTM)} dengan mekanisme \textit{attention} untuk prediksi kadar glukosa darah jangka pendek. Model ini mengombinasikan lapisan \textit{self-attention}, unit \textit{LSTM}, dan lapisan keluaran terhubung penuh untuk memproses deret waktu kadar glukosa. Diadaptasi dari \textcite{Ghimire2024}.}
    \label{gambar:lstm-architecture}
\end{figure}

Gambar~\ref{gambar:lstm-architecture} menunjukkan arsitektur gabungan \textit{LSTM–Attention} sebagaimana dijelaskan oleh \textcite{Ghimire2024}. Model ini terdiri dari tiga bagian utama. Pertama, lapisan \textit{input embedding} dan \textit{self-attention} bertugas mengekstraksi fitur temporal penting dari data kadar glukosa historis. Kedua, lapisan \textit{LSTM cell} menangkap dependensi jangka panjang dan hubungan non-linear antar nilai glukosa, sementara mekanisme gerbang—\textit{forget gate}, \textit{input gate}, dan \textit{output gate}—mengatur aliran informasi antar langkah waktu. Ketiga, lapisan keluaran (\textit{fully connected layer} dan \textit{output layer}) menghasilkan prediksi kadar glukosa masa depan \( G_{t+p} \) berdasarkan masukan deret waktu \( G_{t-L}, G_{t-L+1}, \ldots, G_t \).

Kombinasi antara \textit{attention mechanism} dan arsitektur LSTM terbukti meningkatkan kemampuan model dalam menangkap hubungan temporal yang kompleks dibandingkan LSTM murni \autocite{Ghimire2024}. Pendekatan ini memungkinkan model untuk fokus pada segmen data yang paling relevan untuk prediksi, sehingga meningkatkan akurasi dan stabilitas hasil pada berbagai dataset pasien.

Untuk menilai kinerja model prediktif, digunakan berbagai metrik evaluasi kuantitatif. Dua metrik yang paling umum adalah \textit{Root Mean Square Error (RMSE)} dan \textit{Mean Absolute Error (MAE)}. RMSE mengukur akar rata-rata kuadrat selisih antara nilai aktual dan nilai prediksi, sedangkan MAE mengukur rata-rata selisih absolut keduanya. Rumus matematis dari kedua metrik tersebut diberikan pada Persamaan~\ref{rumus:rmse-mae}.

\begin{equation}
    \text{RMSE} = \sqrt{\frac{1}{n} \sum_{i=1}^{n} (y_i - \hat{y}_i)^2}, \quad
    \text{MAE} = \frac{1}{n} \sum_{i=1}^{n} |y_i - \hat{y}_i|
    \label{rumus:rmse-mae}
\end{equation}

Nilai RMSE dan MAE yang lebih kecil menunjukkan performa model yang lebih baik. Selain metrik statistik, beberapa penelitian juga menggunakan metrik berbasis klinis seperti \textit{Clarke Error Grid (CEG)} untuk mengevaluasi implikasi klinis dari kesalahan prediksi \autocite{Battelino2019}. Dengan kombinasi evaluasi matematis dan klinis ini, model pembelajaran mesin dapat dinilai tidak hanya berdasarkan akurasi prediksi, tetapi juga keamanan penerapannya dalam konteks medis.

Pendekatan berbasis ML seperti LSTM memiliki potensi besar untuk diintegrasikan dengan konsep \textit{Digital Twin}. Model DT dapat menggunakan hasil prediksi ML untuk memperbarui kondisi virtual pasien secara real-time, sementara data simulatif dari DT dapat digunakan kembali untuk melatih model ML, menciptakan sistem pembelajaran adaptif dua arah antara dunia nyata dan digital.

\section{Retrieval-Augmented Generation (RAG) Framework}

\textit{Retrieval-Augmented Generation} (RAG) merupakan pendekatan dalam bidang \textit{Natural Language Processing} (NLP) yang mengombinasikan dua proses utama, yaitu \textit{retrieval} dan \textit{generation}. Pendekatan ini pertama kali diperkenalkan oleh \textcite{Lewis2020} untuk meningkatkan kemampuan model generatif dalam menghasilkan keluaran yang faktual dan berbasis pengetahuan. Tidak seperti model generatif murni yang mengandalkan parameter internal untuk seluruh proses inferensi, RAG memungkinkan model mengakses basis pengetahuan eksternal secara dinamis.

Arsitektur RAG terdiri atas dua komponen utama:
\begin{enumerate}
    \item \textbf{Retriever}, yaitu modul non-parametrik yang bertugas mencari potongan informasi atau dokumen yang relevan dari basis pengetahuan eksternal menggunakan metode pencarian berbasis vektor seperti \textit{Maximum Inner Product Search (MIPS)};
    \item \textbf{Generator}, yaitu model bahasa besar berbasis \textit{sequence-to-sequence transformer} yang menghasilkan keluaran akhir berdasarkan masukan dan konteks hasil \textit{retrieval}.
\end{enumerate}

\begin{figure}[t]
    \centering
    \captionsetup{justification=centering}
    \includegraphics[width=0.95\textwidth]{image/rag_framework_lewis2020.png}
    \caption{Arsitektur \textit{Retrieval-Augmented Generation (RAG)} yang terdiri atas dua komponen utama: \textit{Retriever} untuk mengambil dokumen relevan dari basis pengetahuan eksternal dan \textit{Generator} untuk menghasilkan keluaran berbasis konteks. Model dilatih secara end-to-end sehingga kedua komponen dapat dioptimalkan bersamaan. Diadaptasi dari \textcite{Lewis2020}.}
    \label{gambar:rag-framework}
\end{figure}

Gambar~\ref{gambar:rag-framework} memperlihatkan alur kerja RAG sebagaimana dijelaskan oleh \textcite{Lewis2020}. Proses dimulai dengan kueri masukan (\textit{query encoder}) yang diubah menjadi representasi vektor. Representasi ini digunakan oleh modul \textit{retriever} untuk mencari dokumen relevan dari indeks pengetahuan eksternal. Dokumen yang ditemukan kemudian diteruskan ke modul \textit{generator}, yang memanfaatkan informasi tersebut untuk membentuk keluaran berbasis konteks. Proses pelatihan dilakukan secara end-to-end dengan pembaruan parameter bersama melalui mekanisme \textit{backpropagation}, sebagaimana ditunjukkan oleh panah berlabel “End-to-End Backprop through q and pθ”.

Pendekatan ini memberikan peningkatan signifikan dibandingkan model generatif murni karena mampu:
\begin{itemize}
    \item Mengakses pengetahuan terkini tanpa perlu pelatihan ulang model besar;
    \item Menghasilkan keluaran yang lebih faktual, relevan, dan dapat ditelusuri sumbernya;
    \item Mengurangi fenomena \textit{hallucination} dalam keluaran model bahasa besar.
\end{itemize}

\textcite{Borgeaud2022} memperluas konsep ini melalui integrasi retrieval skala besar, memungkinkan model mengakses triliunan token teks eksternal secara efisien untuk meningkatkan penalaran berbasis pengetahuan. Hasil penelitian tersebut menunjukkan bahwa RAG dapat berfungsi sebagai mekanisme penghubung antara model generatif dan sistem pencarian, menjadikannya salah satu pendekatan utama dalam pengembangan kecerdasan buatan berbasis pengetahuan.

Dalam konteks kesehatan digital, framework RAG berpotensi diintegrasikan dengan \textit{Digital Twin}. Model DT yang biasanya bergantung pada data fisiologis pasien dapat diperluas dengan mengakses basis pengetahuan medis eksternal, seperti literatur ilmiah atau panduan klinis. Melalui integrasi ini, \textit{RAG Digital Twin} dapat memanfaatkan data pasien internal bersama dengan informasi medis eksternal untuk menghasilkan rekomendasi yang lebih adaptif, berbasis bukti, dan kontekstual.

\section{Penelitian Terkait}

Berbagai penelitian telah dilakukan untuk mengembangkan model \textit{Digital Twin} dan sistem prediktif berbasis \textit{machine learning} dalam bidang kesehatan, khususnya manajemen \textit{Diabetes Mellitus}. Beberapa studi juga mulai mengeksplorasi integrasi antara pembelajaran mesin, representasi pengetahuan, dan sistem generatif seperti \textit{Retrieval-Augmented Generation (RAG)}. Tinjauan ini bertujuan untuk mengidentifikasi kemajuan terkini sekaligus menemukan celah penelitian yang relevan untuk konteks Indonesia.

Berdasarkan Tabel~\ref{tbl:penelitian-terkait}, penelitian \textcite{Rad2024}, \textcite{Zhang2024}, dan \textcite{Cappon2024} berfokus pada penerapan \textit{Digital Twin} untuk manajemen diabetes dengan data real-time dari perangkat medis. Meskipun hasilnya menjanjikan, pendekatan tersebut masih memiliki keterbatasan pada aspek biaya, kompleksitas teknologi, dan ketergantungan terhadap infrastruktur \textit{Electronic Health Records} (EHR) yang belum merata di negara berkembang seperti Indonesia.

Sementara itu, penelitian \textcite{Woldaregay2019}, \textcite{li2022time}, dan \textcite{Ghimire2024} menunjukkan bahwa model \textit{machine learning} berbasis deret waktu mampu memprediksi kadar glukosa dengan akurasi tinggi, bahkan menggunakan data simulatif atau publik. Namun, model tersebut masih terbatas pada konteks prediksi numerik dan belum mengintegrasikan pengetahuan klinis eksplisit.

Penelitian \textcite{Lewis2020} dan \textcite{Borgeaud2022} memperkenalkan paradigma \textit{Retrieval-Augmented Generation (RAG)}, yang menggabungkan pencarian informasi eksternal dengan keluaran generatif. Pendekatan ini potensial jika diadaptasi untuk domain medis, di mana sistem tidak hanya memprediksi nilai glukosa tetapi juga menjelaskan dasar fisiologis hasil prediksi tersebut.

Celah penelitian ini menjadi dasar pengembangan \textbf{\textit{RAG Digital Twin Framework for Blood Glucose Prediction}}, yang akan dibahas pada Bab~III.

\begin{longtable}{ | p{2.5cm} | p{3.5cm} | p{3.5cm} | p{3.5cm} | }
\caption{Ringkasan penelitian terkait \textit{Digital Twin} dan prediksi kadar glukosa darah.}
\label{tbl:penelitian-terkait} \\
\hline
\textbf{Peneliti (Tahun)} & \textbf{Pendekatan / Model} & \textbf{Konteks / Dataset} & \textbf{Hasil Utama / Temuan} \\
\hline
\endfirsthead

\caption[]{Ringkasan penelitian terkait \textit{Digital Twin} dan prediksi kadar glukosa darah \textit{(lanjutan)}}\\
\hline
\textbf{Peneliti (Tahun)} & \textbf{Pendekatan / Model} & \textbf{Konteks / Dataset} & \textbf{Hasil Utama / Temuan} \\
\hline
\endhead

\hline \multicolumn{4}{r}{\textit{Bersambung ke halaman berikutnya}} \\
\endfoot

\hline
\endlastfoot

\textcite{Bruynseels2018} & Konseptualisasi \textit{Digital Twin} dalam etika kesehatan & Literatur konseptual & Menyoroti aspek etika dan tanggung jawab sosial dalam penerapan DT di kesehatan. \\
\hline
\textcite{Cappon2024} & \textit{Digital Twin} untuk T1DM & Dataset CGM & Arsitektur tiga tahap (data collection, twin creation, twin use); potensi tinggi namun infrastruktur mahal. \\
\hline
\textcite{Rad2024} & \textit{Patient-Centric Digital Twin} berbasis PHKG & Data EHR dan \textit{wearable} (HL7 FHIR) & RMSE 19.83 mg/dL; memberikan rekomendasi insulin dan diet personal. \\
\hline
\textcite{Zhang2024} & Integrasi DT dengan data multi-omics & Dataset simulatif T2DM & Prediksi progresi T2DM akurat; masih bergantung pada infrastruktur besar. \\
\hline
\textcite{Woldaregay2019} & \textit{Machine Learning} untuk prediksi glukosa & Dataset CGM (real patient) & Model prediksi jangka pendek berbasis regresi non-linear; baseline data-driven. \\
\hline
\textcite{Ghimire2024} & \textit{LSTM + Attention} untuk prediksi glukosa & Beberapa dataset publik & Model dengan \textit{attention} menunjukkan generalisasi lintas dataset yang lebih baik. \\
\hline
\textcite{li2022time} & \textit{Deep Learning (LSTM)} untuk prediksi glukosa & Dataset CGM individual & LSTM outperform model klasik dengan RMSE rata-rata $\leq$ 22 mg/dL. \\
\hline
\textcite{Lewis2020} & \textit{Retrieval-Augmented Generation (RAG)} & Tugas NLP berbasis pengetahuan & Framework retriever–generator meningkatkan akurasi dan relevansi keluaran generatif. \\
\hline
\textcite{Borgeaud2022} & RAG skala besar dengan \textit{retrieval transformer} & Basis data 10 triliun token & Mengurangi \textit{hallucination} dan meningkatkan efisiensi penalaran berbasis pengetahuan. \\
\hline

\end{longtable}
 

\section{Tantangan Implementasi di Indonesia}

Penerapan \textit{Digital Twin} dalam konteks Indonesia menghadapi beberapa hambatan struktural. Pertama, tingkat kesiapan teknologi informasi dan komunikasi (TIK) antar fasilitas kesehatan masih bervariasi \autocite{Aisyah2024}. Kedua, perangkat medis pendukung seperti \textit{Continuous Glucose Monitoring (CGM)} dan pompa insulin masih tergolong mahal dan belum tersedia secara luas \autocite{Ramadaniati2024}. 

Selain itu, penelitian oleh \textcite{Harahap2023} menunjukkan bahwa integrasi data rekam medis elektronik masih terhambat oleh rendahnya adopsi sistem \textit{Electronic Medical Record (EMR)}. Penelitian \textcite{Abodunrin2020} di negara berkembang lain memperlihatkan bahwa kesadaran dan pelatihan tenaga medis terkait sistem digital kesehatan juga menjadi faktor penentu keberhasilan implementasi. 

Oleh karena itu, penerapan \textit{Digital Twin} di Indonesia perlu dimulai dari model sederhana berbasis data simulatif dan pembelajaran mesin yang ringan.