\begin{table}[H]
    \centering
    \caption{Analisis Risiko dan Mitigasi}
    \label{tbl:analisis-risiko}
    \begin{tabular}{|p{3.5cm}|p{4.5cm}|p{5cm}|}
    \hline
    \textbf{Risiko} & \textbf{Dampak} & \textbf{Mitigasi} \\ \hline
    
    Keterbatasan komputasi lokal (Tanpa GPU Diskrit) & 
    Inferensi LLM lokal berjalan lambat atau membebani memori sistem (RAM). & 
    1. Menggunakan model terkuantisasi (4-bit GGUF) yang ringan untuk CPU.\newline
    2. Menggunakan layanan Cloud API untuk modul Generator. \\ \hline
    
    Data input manual tidak konsisten (bolong-bolong) & 
    Akurasi prediksi LSTM menurun karena \textit{missing values}. & 
    Implementasi teknik imputasi data (interpolasi) pada tahap \textit{preprocessing}. \\ \hline
    
    \textit{Overfitting} pada data latih yang kecil & 
    Model gagal memprediksi data baru dengan baik. & 
    Penerapan teknik regularisasi (\textit{Dropout}) dan validasi silang (\textit{Cross-validation}). \\ \hline
    
    Halusinasi pada RAG & 
    Sistem memberikan saran medis yang salah atau tidak relevan. & 
    Membatasi konteks dokumen hanya pada sumber terpercaya (Panduan PERKENI/ADA) dan menurunkan parameter \textit{temperature} LLM. \\ \hline
    
    \end{tabular}
\end{table}