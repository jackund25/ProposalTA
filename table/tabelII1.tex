\begin{longtable}{ | p{2.5cm} | p{3.5cm} | p{3.5cm} | p{3.5cm} | }
\caption{Ringkasan penelitian terkait \textit{Digital Twin} dan prediksi kadar glukosa darah.}
\label{tbl:penelitian-terkait} \\
\hline
\textbf{Peneliti (Tahun)} & \textbf{Pendekatan / Model} & \textbf{Konteks / Dataset} & \textbf{Hasil Utama / Temuan} \\
\hline
\endfirsthead

\caption[]{Ringkasan penelitian terkait \textit{Digital Twin} dan prediksi kadar glukosa darah \textit{(lanjutan)}}\\
\hline
\textbf{Peneliti (Tahun)} & \textbf{Pendekatan / Model} & \textbf{Konteks / Dataset} & \textbf{Hasil Utama / Temuan} \\
\hline
\endhead

\hline \multicolumn{4}{r}{\textit{Bersambung ke halaman berikutnya}} \\
\endfoot

\hline
\endlastfoot

\textcite{Bruynseels2018} & Konseptualisasi \textit{Digital Twin} dalam etika kesehatan & Literatur konseptual & Menyoroti aspek etika dan tanggung jawab sosial dalam penerapan DT di kesehatan. \\
\hline
\textcite{Cappon2024} & \textit{Digital Twin} untuk T1DM & Dataset CGM & Arsitektur tiga tahap (data collection, twin creation, twin use); potensi tinggi namun infrastruktur mahal. \\
\hline
\textcite{Rad2024} & \textit{Patient-Centric Digital Twin} berbasis PHKG & Data EHR dan \textit{wearable} (HL7 FHIR) & RMSE 19.83 mg/dL; memberikan rekomendasi insulin dan diet personal. \\
\hline
\textcite{Zhang2024} & Integrasi DT dengan data multi-omics & Dataset simulatif T2DM & Prediksi progresi T2DM akurat; masih bergantung pada infrastruktur besar. \\
\hline
\textcite{Woldaregay2019} & \textit{Machine Learning} untuk prediksi glukosa & Dataset CGM (real patient) & Model prediksi jangka pendek berbasis regresi non-linear; baseline data-driven. \\
\hline
\textcite{Ghimire2024} & \textit{LSTM + Attention} untuk prediksi glukosa & Beberapa dataset publik & Model dengan \textit{attention} menunjukkan generalisasi lintas dataset yang lebih baik. \\
\hline
\textcite{li2022time} & \textit{Deep Learning (LSTM)} untuk prediksi glukosa & Dataset CGM individual & LSTM outperform model klasik dengan RMSE rata-rata $\leq$ 22 mg/dL. \\
\hline
\textcite{Lewis2020} & \textit{Retrieval-Augmented Generation (RAG)} & Tugas NLP berbasis pengetahuan & Framework retriever–generator meningkatkan akurasi dan relevansi keluaran generatif. \\
\hline
\textcite{Borgeaud2022} & RAG skala besar dengan \textit{retrieval transformer} & Basis data 10 triliun token & Mengurangi \textit{hallucination} dan meningkatkan efisiensi penalaran berbasis pengetahuan. \\
\hline

\end{longtable}
