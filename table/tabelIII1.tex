\begin{table}[H]
\centering
\caption{Kebutuhan Fungsional Sistem}
\begin{tabular}{|p{2cm}|p{12cm}|}
\hline
\textbf{Kode} & \textbf{Deskripsi} \\
\hline
FR-01 & Sistem mampu membaca dan mengelola dataset simulatif seperti \textit{OhioT1DM Dataset} atau \textit{UVA/Padova Simulator}. \\
\hline
FR-02 & Sistem mampu melakukan prediksi kadar glukosa darah jangka pendek menggunakan algoritma \textit{machine learning} seperti LSTM atau model lain yang sesuai. \\
\hline
FR-03 & Sistem menghitung metrik evaluasi seperti \textit{RMSE}, \textit{MAE}, dan \textit{Clarke Error Grid}. \\
\hline
FR-04 & Sistem menyediakan visualisasi grafik perubahan kadar glukosa aktual dan prediksi. \\
\hline
FR-05 & Sistem mampu mensimulasikan perubahan kadar glukosa berdasarkan variasi input (insulin, karbohidrat, aktivitas) sebagai representasi \textit{digital twin}. \\
\hline
FR-06 & Sistem memiliki modul \textit{Retrieval-Augmented Generation (RAG)} untuk mencari informasi medis relevan dari basis pengetahuan eksternal. \\
\hline
\end{tabular}
\end{table}