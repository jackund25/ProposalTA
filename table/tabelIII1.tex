\begin{longtable}{|l|p{10cm}|}
\caption{Kebutuhan Fungsional Sistem}
\label{tbl:functional-requirements} \\
\hline
\textbf{Kode} & \textbf{Deskripsi Kebutuhan} \\ \hline
\endfirsthead

\multicolumn{2}{c}%
{\tablename\ \thetable\ -- \textit{Lanjutan dari halaman sebelumnya}} \\
\hline
\textbf{Kode} & \textbf{Deskripsi Kebutuhan} \\ \hline
\endhead

\hline \multicolumn{2}{r}{\textit{Bersambung ke halaman berikutnya}} \\
\endfoot

\hline
\endlastfoot

FR-01 & Sistem mampu menerima input data harian mandiri (\textit{logbook}) yang mencakup variabel kadar glukosa, asupan karbohidrat, dosis insulin, aktivitas fisik, dan tingkat stres. \\ \hline
FR-02 & Sistem mampu melakukan \textit{preprocessing} untuk menangani ketidakteraturan waktu input (\textit{irregular time-steps}) dari data manual pasien. \\ \hline
FR-03 & Sistem memprediksi kadar glukosa darah jangka pendek menggunakan model \textit{Machine Learning} (LSTM dengan Attention) berdasarkan pola data historis. \\ \hline
FR-04 & Sistem menyediakan fitur simulasi skenario (\textit{What-If Analysis}) yang memungkinkan pengguna mengubah variabel input (misal: mengurangi stres, menambah olahraga) untuk melihat prediksi dampaknya. \\ \hline
FR-05 & Sistem mengintegrasikan modul \textit{Retrieval-Augmented Generation} (RAG) untuk memberikan penjelasan klinis dan rekomendasi aksi yang relevan berdasarkan konteks prediksi. \\ \hline
FR-06 & Sistem menampilkan visualisasi grafik tren glukosa aktual vs prediksi serta hasil simulasi pada antarmuka pengguna. \\ \hline

\end{longtable}